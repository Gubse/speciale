Recall that a decomposable digraph $D$ can be decomposed into a main graph $S$ where $|S|=k$ and $k$ houses $H_1,H_2,\dots , H_k$, where each vertex in $S=\lbrace v_1,v_2\dots ,v_k\rbrace$ is replaced by the house $H_i$ replace $v_i$ and the arcs between the houses is as follows $H_i \rightarrow H_j$ in $D$ if $v_i\rightarrow v_j$ in $S$ remember that for a set $X$ to dominate an other set $Y$ (meaning every vertex in the dominating set dominates every vertex in the dominated set) we denoted it $X \rightarrow Y$. If no arc between $v_a$ and $v_b$ in $S$ then there is no arc between the sets $H_a$ and $H_b$ in $D$. 
The thing about decomposable digraphs is that if there is an arc between $H_i$ and $H_j$ either one of the houses totally dominates the other (ex. $H_i \Rightarrow H_j$) or they dominate each other (ex. $H_i \rightarrow H_j$ and $H_j\rightarrow H_i$).

Decomposable digraphs can be classed by a set of digraphs $\phi$, we say that $D=S[H_1,H_2,\dots ,H_k]$ is \textbf{$\phi$-decomposable} if $D\in \phi$ or if $S\in \phi$. The chioces of $H_i$ for $i=1,2\dots , k$ does not determine anything about the digraph being $\phi$-decomposable but the class of \textbf{totally $\phi$-decomposable} digraphs is where $D$ is $\phi$-decomposable and each $H_i$ is totally $\phi$-decomposable. 
We are going to make two shuch sets of digraphs $\phi_1$ which is the union of semicomplete digraph and acyclic digraph both classes deskribed in \autoref{sec:class} and $\phi_2$ which is the union of semicomplete and round digraphs also deskribed in \autoref{sec:class}.  
\begin{align}
    \phi_1=\text{Semicomplete digraphs}\cup \text{Acyclic digraphs}
    \label{eq:phi1}\\
    \phi_2=\text{Semicomplete digraphs}\cup \text{Round Digraphs}
    \label{eq:phi2}
\end{align}
