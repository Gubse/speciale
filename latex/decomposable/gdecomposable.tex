Recall that a decomposable digraph $D=S[H_1,H_2,\dots H_k]$ can be decomposed into a main graph $S$ (also sometimes called \textbf{quotient} graph), where $|S|=k$ and $k$ houses $H_1,H_2,\dots , H_k$, where each vertex in $S=\lbrace v_1,v_2\dots ,v_k\rbrace$ is replaced by the house ($H_i$ replaces $v_i$).
The arcs between the houses are as follows: $H_i \rightarrow H_j$ in $D$ if $v_i\rightarrow v_j$ in $S$. 
Remember that for a set $X$ to dominate another set $Y$ (meaning every vertex in the dominating set dominates every vertex in the dominated set), we denoted it $X \rightarrow Y$. 
If no arc between $v_a$ and $v_b$ is in $S$, then there is no arc between the sets $H_a$ and $H_b$ in $D$. 
A nice property of decomposable digraphs is that if there is an arc between $H_i$ and $H_j$, either one of the houses totally dominates the other (ex. $H_i \Rightarrow H_j$) or they dominate each other (ex. $H_i \rightarrow H_j$ and $H_j\rightarrow H_i$).

\noindent Decomposable digraphs can be classified by a set of digraphs $\phi$. 
When $D=S[H_1,H_2,\dots ,H_k]$ it is \textbf{$\phi$-decomposable} if $D\in \phi$ or if $S\in \phi$. 
The chioces of $H_i$ for $i=1,2\dots , k$ does not determine anything about the digraph being $\phi$-decomposable, but the class of \textbf{totally $\phi$-decomposable} digraphs is where $D$ is $\phi$-decomposable and each $H_i$ is totally $\phi$-decomposable. 
We are going to make two such sets of digraphs: $\phi_1$, which is the union of semicomplete digraphs and acyclic digraphs both classes described in \autoref{sec:class} and $\phi_2$, which is the union of semicomplete and round digraphs also described in \autoref{sec:class}.  
\begin{align}
    \phi_1=\lbrace \text{Semicomplete digraphs}\rbrace\cup \lbrace \text{Acyclic digraphs}\rbrace
    \label{eq:phi1}\\
    \phi_2=\lbrace \text{Semicomplete digraphs}\rbrace\cup \lbrace \text{Round digraphs}\rbrace
    \label{eq:phi2}
\end{align}

Take these sets $\phi_1$ and $\phi_2$. 
Then for every induced subdigraph of a digraph $D$ where either $D\in \phi_1$ or $D\in \phi_2$, then the induced digraph is in the same set (so if $D\in \phi_1$ the induced subdigraph is in $\phi_1$, same goes for $\phi_2$).
When this is true for a set $\phi$, the set is called \textbf{hereditary}. So both $\phi_1$ and $\phi_2$ are hereditary.
\begin{lemma}
    Let $\phi$ be a hereditary set of digraphs. If a given digraph $D$ is totally $\phi$-decomposable, then every induced subdigraph $D'$ of $D$ is totally $\phi$-decomposable.
    \label{lemma:hereditary}
\end{lemma}
It also turns out that for $\phi_1$ and $\phi_2$, there exists an algorithm that checks whether a digraph $D$ is totally $\phi_i$-decomposable ($i=1,2$).
\begin{thm}~\cite{banggutin}
    There exists an $O(n^2m+n^3)$-algorithm for chekking if a digraph with $n$ vertices and $m$ arcs is totally $\phi_i$-decomposable for $i=1,2$.
    \label{thm:phipoly}
\end{thm}
\noindent $O(n^2m+n^3)$ is clearly a polynomial algortihm. 