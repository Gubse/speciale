This problem is much like the problem we just went through, except instead of linking terminals with vertex disjoint paths, these path only need to be arc disjoint.
Which makes the problem appear more likely in digraphs but also harder to control since there are other checks to go through. \\
Given a set of terminal pairs $(s_1,t_1),\dots ,(s_k,t_k)$, finding arc-disjoint paths between each pair is called the weak $k$-linkage problem.
where a terminal  pair is a source and a sink in the paths of the solution of the linkage problem. \\
The weak linkage problem is also NP-complete because the linkage problem is. 
By vertex splitting, we can make a linkage problem to a weak linkage problem. 
The notation in this chapter is much like in the last: $D$ as the digraph we are examining and $\Pi$ is the set of pairs of terminals, where $k$ is the number of pairs of terminals. 
When we consider the linkage problem for decomposable digraphs, we can have houses with terminals in and some without any terminals. 
The houses contaning no terminals are called \textbf{clean houses}.
Then a terminal pair can either be inside the same houses or in different houses. 
We adopt these definitions from the linkage problem \textbf{internal pairs (and paths)} and \textbf{external pairs (and paths)}.\\
%If a terminal pair is contained inside the same house it is called an internal pair other wise we call the pair external.
%The same with a path if it is fully containd inside a house it is called an internal path other wise it is external.
%you can have an external path for a internal piar if the path go out of the house and in agian.
Since we are focusing on arcs, then letter $F$ will be the main notation of a set of arcs from the digraph $D$, ($F\subseteq A(D)$). 
$F$ is usually used for arcs that we do not want a part of the linkage we are focusing on, mostly because the arcs are already used to link some other pairs. 
Therefore, when focusing on a vertex out- and in-degree compared to the set $F$, it is usually bound by the number of pairs already linked. 
Since the paths do not need to be vertex-disjoint we can end up in the same vertex as another path that links another pair, then we need to somehow ensure that we do not choose the same arc as we did linking the other pair and that is here the set $F$ comes in.  
This is important since deleteing arcs could change the class the digraph belongs to. 
