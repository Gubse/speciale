A locally semicomplete digraph is either round decomposable, semicomplete or neither.
We have in \autoref{sec:locally} called these evil locally semicomplete digraph or just evil.
The semicomplete part is solved from \autoref{thm:semiklink} but the theorem will also be important in this section.
First we will look at the evil semicomplete digraph where we need to recall \autoref{thm:evildecom} $(a)$ where we can see that an evil semicomplete digraph can be partitioned into into maximum 4 semicomplete digraphs $S,D_1',D_2',D_3'$ which lead us to the next theorem.
\begin{thm}~\cite{chudnovskyJCT135}
    For every fixed pair of positive integers c,k there exists a polynomial algorithm for the $k$-linkage problem on digraphs whose vertex set is partinionable into $c$ sets inducing semicomplete digraphs.
    \label{thm:inducesemi}
\end{thm}
Let $c=4$ in \autoref{thm:inducesemi}. 
Then we know from \autoref{thm:evildecom} that every evil locally semicomplete digraph has a polynomial algorithm for the $k$-linkage problem when $k$ is fixed.\\

The remaning class of digraphs inside the class of locally semicomplete digraphs is the class of round decomposable digraphs. 
Recall the class $\phi_2=\lbrace\text{semicomplete digraphs}\rbrace\cup\lbrace\text{round digraphs}\rbrace$ from \autoref{sec:gdecomposable}.
As we did in \autoref{sec:lQuasi}, we will in the end prove that $\phi_2$ is a linkage ejector and since round decomposable digraphs are totally $\phi_2$-decomposable, we would have proven that there exists a polynomial algorithm for them.
To prove that $\phi_2$ is a linkage ejector, we know from \autoref{def:ejector} that it needs 3 algorithms $\mathcal{A}_{\phi_2},\mathcal{B}_{\phi_2}$, and $\mathcal{C}_{\phi_2}$. For the algorithm $\mathcal{B}_{\phi_2}$ we only need it for round digraphs.
\begin{thm}
    For every fixed $k$, there exists a polynomial algorithm to solve the $k$-linkage problem on round digraphs.
    \label{thm:roundklink}
\end{thm}
\begin{proof}
    $D$ is round so let $v_1,\dots ,v_n$ be the round ordering and $\Pi = \lbrace (s_1,t_1), \dots (s_k,t_k)\rbrace$ the set of pairs of terminals.
    Given $j\in [n-1]$, we say that an arc $v_av_b$ is \textbf{over} another arc $v_jv_{j+1}$ if $v_b\in \lbrace v_j+1,\dots v_{a-1}\rbrace$. 
    We are now going to show that for a $(s_i,t_i)$-path, it only needs to use one arc over $v_jv_{j+1}$. 
    Let us assume this is not the case and that the $(s_i,t_i)$-path uses two arcs over $v_jv_{j+1}$. 
    Call these two arcs $u_1w_1$ and $u_2w_2$. 
    There are four ways these vertices can be placed in relation to each other in the ordering and still be arcs over $v_jv_{j+1}$. See figure \textcolor{red}{lav figur}.
    Let us say without loos of generality that the $(s_i,t_i)$-path first the arc $u_1w_1$ and then later $u_2w_2$.
    We can in all cases of constellations of the vertices mentioned in \textcolor{red}{ref figur blalbalbal} make the $(s_i,t_i)$-path shorter by use of other arcs. 
    If we are in either case 1 we can make the path shorter by using the arc $u_1u_2$ and therefore not need to use the arc $u_1w_1$, since $u_1u_2$ is not an arc over $v_jv_{j+1}$ the $(s_i,t_i)$-path only uses one arc over.
    If we instead have the constellation in case 2, case 3 and case 4, we can use the arc $u_1w_2$ which is an arc over $v_jv_{j+1}$, but it means that the path does not use any of the two over arcs $u_1w_1$ or $u_2w_2$. 
    This proves that any $(s,t)$-path only needs to use max one arc over $v_jv_{j+1}$.
    Hence each path in the $k$-linkage only uses maximum $k$ arcs over $v_jv_{j+1}$.
    We also know that deleting all arcs over $v_jv_{j+1}$, we get an acyclic digraph and from \autoref{thm:acyclicklink} that we can solve the $k$-linkage problem in polynomial time on this digraph.\\
    Since the digraph is not acylic, some of the paths can and may need to use an arc over $v_jv_{j+1}$ so we make a combination of some of the pairs, not nessesary all in order, so we rename the $h$ chosen pairs $(s_{i_1},t_{i_1}),\dots ,(s_{i_h},t_{i_h})$ where $0\leq h\leq k$ where $i_k$ is a function mapping $i_1$ to $z\in [k]$ the first chosen piar of the $k$ terminal piars. \\
    These are the pairs we predict uses the arcs $\lbrace u_1w_1,\dots ,u_hw_h\rbrace$ which are all arcs over $v_jv_{j+1}$.
    Then we construct $D'$ by deleting all arcs over $v_jv_{j+1}$ then we have the $2h$-linkage for the pairs $(s_{i_1},u_1),(w_1,t_{i_1}),\dots ,(s_{i_h},u_h),(w_h,t_{i_h})$ and the rest of the original pairs $k-h$-linkage, then we use the algorithm for acyclic digraphs and solve the $k+h$-linkage on $D'$. Do this for all combinations of $h$ pairs. 
    If there exists a $k$-linkage in $D$, there exists some $k+h$-linkage in $D'$ for some combination of $h$ pairs.
    When the $k+h$-linkage is found, we add the arcs $u_1w_1,\dots , u_hw_h$ to the linkage and create a path $P_{i_j}$ from the path $(s_{i_j},u_j)$-path then the arc $(u_j,w_j)$ and at last  $(w_j,t_{i_j})$-path which is a $(s_{i_j},t_{i_j})$-path.
    This creates the $k$-linkage for $D$. 
\end{proof}
The last algortihm will be in the proof of the next theorem which will end the part about round decomposable digraphs.
\begin{thm}
    For every fixed $k$, there exists a polynomial algorithm to solve the $k$-linkage problem on round decomposable digraphs.
\end{thm}
\begin{proof}
    First we know from \autoref{sec:locally} that round-decomposable digraphs are totally $\phi_2$-decomposable. 
    So if $\phi_2$ is a linkage ejector then we can use \autoref{alg:phi} to find the $k$-linkage of a round decomposable digraph. 
    This means all that is left to prove is that $\phi_2$ is a linkage ejector, for this we need to prove that $\phi_2$ is closed with respect to blow-ups.
    The semicomplete digraphs we know from the proof of \autoref{thm:phi1ejector} that we can blow it up with a transitive tournament. 
    A transitive tournament is also a round digraph, which means that we can blow up a vertex in a round digraph and it is still round. A short argument of this:\\
    We know from the proof of \autoref{thm:phi1ejector} that a transitive tournament is acyclic meaning we have an acyclic ordering of the vertices $v_1,v_2,\dots , v_n$. 
    Since it is a tournament, we know taht a vertex $v_i$ is dominated by all vertices before in the acylic ordirng $v_1,\dots v_{i-1}=N^-(i)$ and dominate $v_{i+1},\dots v_n=N^+(i)$. 
    This is true for all $i\in[n]$. 
    Thus the acyclic ordering is also the round ordering.\\
    So now we know that $\phi_2$ is closed w.r.t. blow-ups as long as it blows-up to a transitive tournament. 
    The algorithm $\mathcal{A}_{\phi_2}$ is covered by \autoref{thm:phipoly}.
    Now for algorithm $\mathcal{B}_{\phi_2}$ we have for the semicomplete digraphs a polynomial algorithm for the $k$-linkage problem by \autoref{thm:semiklink} and for round digraphs we have the algorithm for the $k$-linkage problem by \autoref{thm:roundklink}, thus combining these theorems we have $\mathcal{B}_{\phi_2}$. 
    The last algorithm $\mathcal{C}_{\phi_2}$ delete and add arcs from each $M_i$ in a decomposition $R[M_1,M_2,\dots , M_r]$ so it becomes a transitive tournament.  
    This proves that the class $\phi_2$ a is likage ejector.
\end{proof}
Now we have an algorithm for all locally semicomplete digraphs and therefore to end this section we have the following theorem:
\begin{thm}
    For every fixed $k$, there exists a polynomial algorithm to solve the $k$-linkage problem on locally semicomplete digraphs.
\end{thm}
