Locally semicomplete digraph can be round-decomposable it turns out that we can from the independece number $\alpha (D)$ tell wether a digraph is round-decomposable or not.
Recall independence number from \autoref{sec:digraph}. The theorem below is from \cite{bangJGT77} where we omits some part of it since we have it statet elsewhere in the thises.
\begin{thm}~\cite{bangJGT77}
    A locally semicomplete digraph $D$ havinng idependece number $\alpha (D)$ at least 3 is round decomposable with a unique round -decomposition. 
    \label{thm:independenceround}
\end{thm} 
This means when considering all other locally semicomplete digraphs it has an independence number $\alpha (D) \leq 2$ which means for all not round-decomposable locally semicomplete digraphs we can use the algorithm in \autoref{thm:independencepoly} to solve the weak $k$-linkage problem when $k$ is fixed.
\begin{thm}
        For every natural number $\alpha$ the weak $k$-linkage problem is polynomial for every fixed $k$, when we consider digraphs with independence number at most $\alpha$.
    \label{thm:independencepoly} 
\end{thm}
For solving the weak $k$-linkage problem in locally semicomplete digraphs we now only need to find a polynomial algorithm for the round-decomposable once.
Before going into this we have to introduce something called the cutwidth. This definition of cutwidth is inspired by the describtion of the cutwidth in \cite{bangJGT77}.\\
Given a digraph $D$ and an ordering of the vertices $O=v_1,\dots,v_n$ we say that the ordering $O$ has a \textbf{cutwidth} at most $\theta$ if $\forall j\in {2,3, \dots n}$ there are at most $\theta$ arcs $u,v$ with $u\in \lbrace v_1,\dots ,v_{j-1}\rbrace$ and $v\in \lbrace v_j,\dots ,v_n\rbrace$ 
\textcolor{red}{inset figur som viser cutwidth $\theta$ for givet ordering}.
Say we have another ordering $O'$of the same digraph $D$, if $O'$ has a cutwidth at most $\theta$ for all possible orderings $O'$ of $D$, then $D$ is said to have a \textbf{cutwidth} at most $\theta$. \\
The minimum natural number $\theta$ such that $D$ has a cutwidth at most $\theta$, we call $\theta$ \textbf{the cutwidth} of $D$.
When we know the cutwidth of the digraph we can solve the weak $k$-linkage problem for those i polynomial time.
\begin{thm}~\cite{bangJGT77}
    For every natural number $\theta$ the weak $k$-linkage problem is polynomial for every fixed $k$, when we consider digraphs with cutwidth at most $\theta$.
    \label{thm:cutwidthklink}
\end{thm}
For the rest of this section  \textbf{interval} will be used with respect to the round ordering of a round digraph. An \textbf{interval} is a subset of vertices $v_iv_{i+1}\dots v_{j-1}v_j$ where the vertices is consecutive compared to the round ordering. In this case the intervals left and right endpoints is $v_i$ and $v_j$ respectively.
From a round digraph $D$ we are going to construct another digraph called the \textbf{compression of $D$ with respect to $\Pi$} and is denoted $D_{\Pi}$. \\
We will Now introduce disjoint intervals $I_1,\dots ,I_l$ where all terminals are containt in there union ($\Pi \subseteq \bigcup_i ^l I_i$) and the left endpoint of each $I_i$ is a terminal.
Also the next $6k$ vertices on the left of the interval $I_i$ are not terminals.
The next $6k$ vertices on the right of the interval are not terminals either this is true for all $i\in [l]$. 
let $I_1$ be the interval which left endpoint is a terminal with the lowest possible number in the round ordering that adhere the properties of the intervals endpoints.
This condition enforces uniqeuness.\\
This can be done unless the digraph is smaller than $12k^2$.
How we find these intervals will be deskribed later in this section.
First we want to introduce $L_i$ and $R_i$ which are both intervals of the round ordering and $L_i$ is the $3k$ vertices left of $I_i$ and $R_i$ is the $3k$ vertices right of $I_i$.
Now we have the rest of the vertices that are not in any interval and we define $W_i$ to be the interval of the vertices between $R_i$ and $L_{i+1}$.
See figure in \cite{bangJGT77} page 103.\\
Now the compression of $D$ with respect to $\Pi$ is the digraph obtaint from $D$ by contracting $W_i$ and if nessesary we delete arcs such that only $k$ multiple arcs is left (the maximum multiplicity of an arc is $k$).

We want to show that finding a weak $\Pi$-linkage in a round digraph $D$ you can just as well find a weak $\Pi$-linkage in its compression with respect to $\Pi$. 
This can only be done for round digraph with cutwidth at least $\Theta =k(6k+36k^2(2k+1)^2)$.
Since the cutwidth is so large we can clearly see that the size of $D$ is not smaller than $12k^2$ so we can construct $D_{\Pi}$.

\begin{lemma}
    Let $D$ be a round digraph with round ordering $O$ and cutwidth at least $\Theta$. 
    Let $\Pi$ be a list of terminal pairs. 
    $D$ has a weak $\Pi$-linkage if and only if its compression with respect to $\Pi$, $D_{\Pi}$, has a weak $\Pi$-linkage.
    \label{lemma:iwanttoprove}
\end{lemma} 

The construction of $D_{\Pi}$ is based on the intervals $I_i$. 
For the first interval $I_1$ we find the first terminal $\tau$ where any of the $6k$ vertices left of $\tau$ are not terminals.
We now make $\tau$ the left endpoint of $I_1$ and look at the $6k$ vertices to the right if they contain another terminal $\tau '$ we let every vertex uptill $\tau '$ including $\tau '$ be in $I_1$ and look right on the next $6k$ vertices from $\tau '$ if it contains a terminal include it in $I_1$ as we did with $\tau '$ if no twe have $I_1$ and we know that the next terminal $\tau *$ right of $I_1$ has at least $6k$ vertices to the left that are not terminals. We now make $\tau *$ the left endpoint of $I_2$ then we do with $I_2$ as we did with $I_1$.
We keep doing this untill all terminals is a part of an interval $I_i$.
Then we can easily construct $L_i$ and $R_i$ from $I_i$ and from this we can find $W_i$ if it exists (is not empty) do this forall $i\in [l]$. then we constract $W_i$ and we now have constructed $D_\Pi$. \\
In this thesis we are focusing on the decomposable digraphs and we can define a compression for the round decomposable digraphs too. As the compression for round digraphs the compression of round decomposable digraphs is both defined in $\cite{bangJGT77}$ page 102 - 105.
So we assume $D=R[H_1,\dots ,H_r]$ is round decomposable. 
Then we contract the clean houses so we now have $D'=R'[H'_1,\dots ,H'_r]$ which is the digraph after the contraction. 
The only difference between $R'$ and $R$ is the multiplicity of the arcs, we will construct $\Pi '$ from $\Pi$ where for each pair, $(s_i,t_i)\in \Pi$ where $s_i\in H_z$ and $t_i\in H_q$, we make a pair $(v_z,v_q)\in \Pi'$ where $v_z,v_q\in V(R')$.
We now make a compression of $R'$ with respect to $\Pi'$, $R'_{\Pi'}$. 
Let $v_{j_1},\dots ,v_{j_p}$ be the vertices of the compression $R_{\Pi'}$.\\
Now we define the compression of $D$ with respect to $\Pi$ to be the digraph $D\Pi=R'_{\Pi'}[H'_{j_1},\dots ,H'_{j_p}]$.
The intervals $I_j$ are the only once with terminals in and therefore the only intervals of $R'_{\Pi'}$ that have some blown-up vertices in $D_\Pi$.
We know from $\autoref{lemma:cleanhouse}$ that we can contract the clean houses and in the prove of \autoref{lemma:iwanttoprove}, which can be found in $\cite{bangjgt77}$ page 103-104, that the path are split up in $(s_i,\sigma _i)$-path, $(\sigma_i,\tau_i)$-path $(\tau_i,t_i)$-path that obviouse joined together is an $(s_i,t_i)$-path. 
The $(\sigma_i,\tau_i)$-path is not inside any $I_b$ interval and follow therefore by lemma 5.5 in \cite{bangJGT77}. 
Both the $(s_i, \sigma_i)$-path and the $(\tau_i,t_i)$-pathdo not use the property of $I_b$ being round and can therefore be linked in the same way for $D_\Pi=R'_{\Pi'}[H'_{j_1},\dots ,H'_{j_p}]$. 
Which brings us to this lemma.
\begin{lemma}
    Let $D$ be a digraph of the form $D=R[H_1,\dots H_r]$, where $R$ is round and has cutwidth at least $\Theta$. Let $\Pi$ be a list of piars of terminals. $D$ has a $\Pi$-linkage if and only if $D_\Pi$, has a $\Pi$-linkage.
    \label{lemma:compressiondecom}
\end{lemma}
Now we can use all this to prove that $\phi_2$ which is defined in \autoref{sec:gdecomposable} is bombproof and recall that round-decomposable digraphs is totally $\phi_2$-decomposable.
\begin{lemma}
    The class $\phi_2$ is bombproof
\end{lemma}
\begin{proof}
    For $\phi_2$ to be bombproof we need to find $\mathcal{A}_{\phi_2}$ which we have from \autoref{lemma:phipoly}.
    we have already proven the existens of $\mathcal{B}_{\phi_2}$ if $R\in\phi_2$ is semicomplete for this see the prove of \autoref{thm:phi1bomb}.
    Therefore assume that $R$ is round we  want to show that the weak $k$-linkage problem is polynomial on $R(c)$ for a positive integer $c$.
    Let a digraph $D\in R(c)$. Now recall $\Theta = k(6k+36k^2(2k+1))^2$ then when $R$ is round we will base the prove on two cases one where $R$ has a cutwidth at least $\Theta$ and another where $R$ has cutwidth at most $\Theta$.\\
    In both cases we have $D=R[H_1,\dots ,H_r]$ where at most $c$ of the $H_i$ houses has $|V(H_i)|>1$ and $R$ has an ordering $O$, $v_1,\dots ,v_r$ where $H_i$ in $D$ corresponds to $v_i$ in $R$.
    \begin{itemize}
        \item[Case 1] When a digraph $D=R[H_1,\dots ,H_r]$ with size $|V(R)|\geq 12k^2$ we can create a compression of $R$ with respect to some $\Pi*$ created from $\Pi$, and therefore we can construct the compression of $D$ with respect to $\Pi$, $D_{\Pi}$.
        As we know from the way we constructed $D_\Pi$ the size is only depending on $c$ and $k$, since $R_{\Pi*}$ is has a size depending on $|\Pi*|\leq k$ and we blow up at most $k$ vertices to a size at most $c$. 
        Since $c$ and $k$ are both fixed naturel numbers we use a brute-force algorithm (an algorithm that checks all possibilities) to solve the weak $\Pi$-linkage problem on $D_\Pi$ and from \autoref{lemma:compressiondecom} $D$ has a weak $\Pi$-linkage if and only if $D_\Pi$ has a weak $\Pi$-linkage.\\
        Since $c$ and $k$ are fixed the brute-force algorithm is polynomial and the construction of $D_\Pi$ is also polynomial.
        \item[Case 2] $R$ has a cutwidth at most $\Theta$ for the round ordring $O$ so for $D=R[H_1,\dots ,H_r]$ we construct an ordering $O'$ where for every $u\in H_i$ and $z\in H_j$ with $i\neq j$ we have that $u < z$ in $O'$ if $v_i < v_j$ in $O$.
        The ordering of the vertices inside a house $H_l$ is not important for the prove.
        Now cutwidth $\theta'$ of $O'$ is at most $k(c^3+c^2\cdot \Theta)$.
        To calculate this we know that there is at most $c$ houses $H_i$ where $|V(H_i)|>1$ these houses has size at most $c$. 
        There is at most $c^2$ arcs inside a house with possible multiplicity $k$ since we are not interested in more. So for arcs inside the houses that can contribute to the cutwidth is $c^2\cdot k\cdot c$.
        The other arc that can contribute to the cutwidth $\theta'$ is the arcs between the houses $H_i$ and $H_j$ where $v_i<v_j$ in $O$. 
        The arcs between two such given houses is $c\cdot c\cdot k$ since both has a size on maximum $c$ and the multiplicity of these arcs is at most $k$ we can not find more than $\Theta$ cases of such two houses since they represent vertices of $R$ with cutwidth at most $\Theta$.
        So $\theta'\leq c^3\cdot k+c^2\cdot \Theta \cdot k=k(c^3+c^2\cdot \Theta)$.
        Therefore we can use the algorithm from \autoref{thm:cutwidthklink} to solve the $k$-linkage problem of $D=R[H_1,\dots ,H_r]$ ($R(c)$).
    \end{itemize}
    Thus we have found $\mathcal{B}_{\phi_2}$.
\end{proof}
As mensioned above and proved in \autoref{sec:locally} round-decomposable digraphs is totally $\phi_2$-decomposable and we have just proved that $\phi_2$ is bombproof so by the algortihm \autoref{alg:weakphi} for bombproof classes every round-decomposable digraph now have a polynomial algorithm to solve the weak $k$-linkage problem.
\begin{thm}
    For every fixed $k$ there exists a polynomial algorithm for the weak $k$-linkage problem for round-decomposable digraphs.
\end{thm}
This ends the part for round-decomposable digraph and in the begining of this section we proved that all other locally semicomplete digraphs than the round-decomposable once have a polynomial algorithm for the weak $k$-linkage problem. We can now state this.
\begin{thm}
    For every fixed $k$ there exists a polynomial algorithm for the weak $k$-linkage problem for locally semicomplete digraphs.
\end{thm}