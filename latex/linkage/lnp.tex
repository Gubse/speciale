Given a digraph $D$ and two distinct vertices $s$ and $t$, we want to make a path from $s$ to $t$ denoted by $P$. 
Recall that in this case $s$ will be the source of $P$ and $t$ the sink.
Now let $s_1,t_1,\dots ,s_k,t_k$ be distinct vertices of $D$, then the \textbf{$k$-linkage} problem is to decide if there exist paths $P_1,\dots ,P_k$ linking the vertices so $P_i$ link the pair $(s_i,t_i)$ $\forall i\in [k]$ and each path $P$ has to be vertex disjoint from the others. 
This problem is also sometimes just called \textbf{the linkage problem}.
The problem is actually NP-complete already when $k=2$, so we will be focusing on the problem when $k$ is fixed.
%This we sould be able to do easily if one exists, but when adding two extra distict vertices $u$ and $v$ not nessesarily distinct from $s$ and $t$ and we want a path $Q$ between $u$ and $v$ distinct from the path $P$ the problem suddenly become NP-complete.
The vertices $s_1,t_1,\dots , s_k,t_k$ are called \textbf{terminals} and $(s_1,t_1);\dots ;(s_k,t_k)$ are called \textbf{terminal pairs}.
The notation for this problem in this thesis would be using $k$ as the natural number of pairs of terminals, and the set of these terminals is denoted $\Pi=\lbrace (s_1,t_1);\dots ;(s_k,t_k)\rbrace$. 
As we have done up till now we will still use $D$ as the main digraph we are looking at unless anything else is specified. 
$L$ is used as a collection of paths $P_1,\dots , P_l$. 
If $L$ is the solution to our linkage problem it means $l=k$ and the path $P_i$ links the pair $(s_i,t_i)$ for all $i\in [k]$.
If $L$ upholds the above conditions we say that $L$ is a $\Pi$-linkage, or $L$ is the linkage of $(D,\Pi)$.\\
Recall that a quasi-transitive digraph is constructed by either a transitive acyclic digraph or semicomplete digraph as the quotient of the decomposition. 
And for these two classes of digraphs, we can solve the $k$-linkage problem in polynomial time for a fixed $k$. 
Fixed $k$ means that for an algorithm given a digraph and a natural number $k$ can solve the $k$-linkage problem in polynomial time (it is possible that the algorithm needs more information). 
So when calculating the running time $k$ is treated as a constant.
When $k$ is not fixed then it is already NP-complete for tournaments.
Since tournaments are a very strict class, we will only focus on when $k$ is fixed.
