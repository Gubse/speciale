Given a digraph $D$ and two distinct vertices $s$ and $t$ we want to make a path from $s$ to $t$ denoted this $P$. 
Recall that in this case $s$ will be the source of $P$ and $t$ the zink.
This we sould be able to do easily if one exists, but when adding two extra distict vertices $u$ and $v$ not nessesarily distinct from $s$ and $t$ and we want a path $Q$ between $u$ and $v$ distinct from the path $P$ the problem suddenly become NP-complete.
This prolem is what we call the \textbf{2-linkage} problem, we can replace 2 with an arbitrary number $k$ and we then call it the \textbf{$k$-linkage} problem or just \textbf{the linkage problem}. 
the vertices $s$, $t$, $u$ and $v$ are called \textbf{terminals} and $(s,t)$, $(u,v)$ are called \textbf{terminal pairs}.
\begin{thm}
    \textcolor{red}{linkage NP-complete}
\end{thm}
\begin{proof}[skezt]
    \textcolor{red}{blblablbalablablabala}
\end{proof}
The notation for this problem in this thesis would be using $k$ as the natruel number of pairs of terminals, and the set of these terminals is denoted $\Pi=\lbrace (s_1,t_1);\dots ;(s_k,t_k)\rbrace$. 
As we have done optil now we will still use $D$ as the main digraph we are looking at unless anything else is specified. 
$L$ is used as a colloction of paths $P_1,\dots , P_l$ if $L$ is the solution to our linkage problem it means $l=k$ and and the paths $P_i$ links the pair $(s_i,t_i)$ for all $i\in [k]$.
If $L$ upholds the above conditions we say that $L$ is a $\Pi$-linkgae, or $L$ is the linkage of $(D,\Pi)$.\\
Recall that a quasi-transitive digraph is build up by either a transitive acyclic digraph or semicomplete digraph as the quotient of the decomposition. 
And for these to classes of digraph we kan solve the $k$-linkage problem in polynomial time for a fixed $k$. With fixed $k$ there means that an algorithm given a digraph and a naturel number $k$ can solve the $k$-linkage problem(it is possible that the algorithm needs more information). When $k$ is not fixed then it is already NP-complete for tournaments, since tournaments is a very strict class we will only focus on when $k$ is fixed.
