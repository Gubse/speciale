A \textbf{Tournament} is a digraph in which the underlying graph is complete. 
So in a complete graph of order 5, any orientation of the edges concludes in a tournament.
Strong digraphs are also in themselves a classification of digraphs. 
Classes of digraphs can overlap each other or fully contained in each other, like tournaments is fully contained in the class called semicomplete digraphs.
A \textbf{semicomplete} digraph is where the underlying graph is a complete multigraph, there can be some multiple edges in between the same pair of vertices in the underlying graph (parallel edges). 
Since the class called semicomplete digraphs contains all digraphs where the underlying graph is a complete multigraph, it clearly also contains the graph with only one arc between every pair of vertices (Tournaments).
A digraph is \textbf{complete} if for every pair of vertices $a,b\in V$, the arc $(a,b)$ and $(b,a)$ are present in the graph. \\
If you can split the vertices of a graph $V$ into two sets of vertices $A$ and $B$ such that $A\cup B=V$, and there are no arcs inside these sets, then we classify this as an \textbf{bipartite} digraph. 
This means all arcs in the graph are in the form $(a,b)$ or $(b,a)$ for all $a\in A$ and $b\in B$. 
The sets $A$ and $B$ are called the partites of $D(V,A)$. 
The underlying graph of a bipartite digraph is also called bipartite since there are no edges inside $A$ or $B$.
%If there exists more then two of these partite sets we call the digraphs \textbf{multipartite}, since there is multiple partite sets in the graph, bipartite sets $\subset$ multipartite. \\
A much used type of digraph is an \textbf{acyclic} digraph. 
It is a digraph where there exists an ordering of the vertices $V={v_1, v_2,\dots , v_n}$ where the arcs in the digraph are $(v_i, v_j)$ where $i<j$ for all $(v_i, v_j)\in A$. 
This ordering is called an \textbf{acyclic ordering}, and there can be many of these orderings in the same digraph.
This ordering can also be used to order strong components in a non-strong digraph, such that the ordering of the components $C_1,C_2,\dots C_k$ is an acyclic digraph when contracting the components into $k$ vertices. 
When classifying digraphs, there are several ways of doing this, like \textbf{transitive} digraphs which are digraphs where for all vertices $a,d,c\in V$ where the arc $(a,b)$ and $b,c$ is present in the digraph ($\in A$), the arc $(a,c)$ has to be a part of $A$ too. 
Using the same kind of classification, there are digraphs which are \textbf{Quasi-transitive}, which is for all vertices $a,d,c\in V$ where the arc $(a,b)$ and $b,c$ is present in the digraph ($\in A$, $a$ and $c$ has to be adecent by at \textit{least} one (more arcs in between are also allowed) arc in either direction ($(a,c)$ or $(c,a)$)). These graphs are going to be mentioned a lot in this thises since the graph is also what we call \textbf{decomposable}.\\
A \textbf{Decomposable} digraph $D=S[H_1,H_2,\dots H_s]$ is a digraph $D$ that can be decomposed into $H_1,H_2, \dots , H_s$ \textbf{houses} and a digraph $S$ denoted as the \textbf{quotient} digraph where $|V(S)|=s$. 
Let $S$ be the quotient digraph of $D$ where $V(S)=\lbrace s_1,s_2,\dots,s_s\rbrace$. 
If each $s_i$ is replaced by the digraph $H_i$ $i=1,2,\dots,k$, we have the digraph $D$, where $H_i\rightarrow H_j \in D$ if $s_i\rightarrow s_j\in S$. 
This is called a \textbf{composition} of $S$ or a \textbf{decomposition} of $D$.
This is the class of digraphs we are focusing on in this thesis. 
If all the houses are independent sets, we call $D=S[H_1,H_2,\dots ,H_k]$ the extension of $S$. 
If $S$ is a semicomplete digraph, we call the extension of these \textbf{extended semicomplete} digraph.
Like we already mentioned, Quasi-transitive digraphs are decomposable but we have several classes that are decomposable, and another class of digraphs that we are giong to cover in this dissertation are \textbf{locally semicomplete} digraphs.\\
First, we introduce \textbf{locally in-semicomplete} digraphs where for every in-neighboor of a vertex, $x\in V$ has to be adjacent. 
This has to be true for all $x\in V$ ($x\cup N^-(x)$ induces a semicomplete digraph $\forall x\in V$). 
When $x\cup N^+(x)$ for all vertices in $D$, it classifies as \textbf{locally out-semicomplete} digraphs. 
Respectively, it is called an out-locally semicomplete digraph if $\forall x\in V$, $N^+(x)$, has to be adjacent. 
If a digraph is both locally in-semicomplete and locally out-semicomplete, it is called a \textbf{locally semicomplete} digraph. 
Why both quasi-transitive digraphs and some locally semicomplete digraphs are decomposeble will be described in \autoref{sec:gdecomposable}.\\
The last class of digraph that are important for this thesis are the round digraphs. 
A digraph is called a \textbf{round} digraph if there exists an ordering of the vertices $v_1,v_2,\dots,v_n$ such that for all $v_i$, $N^+(v_i)={v_{i+1},v_{i+2},\dots ,v_{i+d^+(v_i)}}$ and $N^-(v_i)={v_{i-d^-(v_i)},v_{i-(d^-(v_i)-1)},\dots ,v_{i-1}}$.




