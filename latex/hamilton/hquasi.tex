First of all, we have to recall \autoref{thm:quasidecom} since it is the key theorem to solve the Hamilton cycle problem in polynomial time. \\
Remember that a condition for a digraph to be Hamiltonian is that it need to be strong, so for finding a Hamilton cycle in a quasi-transitive digraph, we are not interested in the non-strong digraphs. 
Leaving only the strong quasi-transitive digraphs with decomposition $S[Q_1,\dots Q_s]$ from \autoref{thm:quasidecom}. 
The given decomposition of a strong quasi-transitive digraph has a semicomplete digraph as the quotient. This is why we need some insight to these before the main solution in this subsection can be proven. 
Another composition of semicomplete digraphs is the extension of these, called extended semicomplete digraphs. 
An extension of a digraph is a composition of the given digraph $S$ where the houses of the composition is either a single vertex or independence sets. \\
Before we explain when we can find a Hamilton cycle in strong quasi-transitive digraphs, we need to recall what a cycle factor is. 
From \autoref{sec:digraph}, we shortly explained that a cycle factor is when we can find $C_1,\dots C_k$ cycles in $D$ contaning all vertices of $D$. 
\begin{thm}~\cite{gutinMNN2}
    An extended semicomplete digraph $D$ is Hamiltonian if and only if $D$ is strong and contains a cycle factor. One can check whether $D$ is Hamiltonian and construct a Hamilton cycle of $D$ (if one exsists) in time $O(n^{2.5})$.
    \label{thm:extended}
\end{thm}
\begin{thm}~\cite{bangJGT20}
    A strong quasi-transitive digraph $D$ with a canonical decomposition \\$D=S[Q_1\dots, Q_s]$ is Hamiltonian if and only if it has a cycle factor $\mathcal{F}$ such that no cycle of $\mathcal{F}$ is a cycle of some $Q_i$.
    \label{thm:qhcycle}
\end{thm}
\begin{proof}
    Since a Hamiltonian cycle needs to cover all vertices in a digraph, we know that it must cross every $Q_i$. 
    Moreover the Hamilton cycle is a cycle factor not fully contained in any $Q_i$. 
    So we only need to show that if we have a cycle factor $\mathcal{F}$, where no cycle is in any $Q_i$, then $D$ is Hamiltonian. $\forall i$ $V(Q_i)\cap \mathcal{F}=\mathcal{F}_i$, there can not be any cyrcle in this and since every vertex is in $\mathcal{F}$, all vertices in $Q_i$ must be contained in $\mathcal{F}_i$ and there is no cycle contained in $\mathcal{F}_i$ which makes it a path factor of $Q_i$.\\
    \textcolor{red}{Figure here}\\
    For all paths in $\mathcal{F}_i$, we make a path contraction. 
    After contraction or before we delete the remaining arcs if this is done before its the arcs going from the end of a path to a beginning of another path. 
    This action will make $Q_i$ an independent set $\forall i\in [s]$. 
    Since $S$ is a semicomplete digraph, the digraph $D$ would then because of the independence of each $Q_i$ after the path contractions be an extended semicomplete digraph $S'$. 
    Since we have only made path contractions along the cycles in the cycle factor of $D$ and not deleted any arcs that are a part of the cycle factor, $S'$ contains a cycle factor. 
    Then by \autoref{thm:extended}, we know that $S'$ contains a Hamilton cycle. 
    Adding the deleted arcs in each $q_i$ does not change the fact that it contains a Hamilton cycle. We now insert the paths we contracted instead of a node, this makes the cycle longer but it still contains every vertex in the digraph, given a Hamilton cycle in $D$.
\end{proof}

A Hamilton path does not have the same condition for a digraphs to be strong meaning we are also interested in the non-strong quasi-transitive digraphs $T[H_1,\dots ,H_t]$. 
The next theorem is proven in much the same as \autoref{thm:qhcycle}. 
\begin{thm}~\cite{bangJGT20}
    A quasi-transitive digraph $D$ with at least two vertices and with canonical decomposition $D=R[G_1,G_2,\dots , G_r]$ is traceable if and only if it has a $1$-path-cycle factor $\mathcal{F}$ such that no cycle or path of $\mathcal{F}$ is completely in some $D\left< V(G_i)\right>$.
\end{thm}
The proof of \autoref{thm:qhcycle} shows that there is a correlation between the cycle factor of $D$ a quasi-transitive digraph and the path cover of $Q_i$ and therefore we have correlation between the path cover of each $Q_i$ and the hamilton path or cycle. to further explore that we have the following theorem.
but first we need to establish some notation, $pc(D)$ is the path covering number of a digraph $D$. 
\begin{thm}
    Let $D$ be a strong quasi-transitive digraph with canonical decompostion $D=S[Q_1,Q_2,\dots,Q_s]$.
    Let $n_1,\dots n_s$ be the size of the digraphs $Q_1,Q_2,\dots ,Q_s$ respectively. 
    Then $D$ is hamiltonian if and only if the extended semicomplete digraph $S'=S[\overline{K}_{n_1},\dots ,\overline{K}_{n_s}]$ has a cycle subdigraph which covers at least $pc(Q_j)$ vertices of $\overline{K}_{n_j}$ for every $j=1,2,\dots , s$.
\end{thm}
\begin{proof}
    Supose $D$ has a hamilton cycle $H$.
    $V(Q_j)\cap H$ is a $k_j$-path factor $\mathcal{F}_j$ of $Q_j$. 
    $pc(Q_j)$ is the minimum number of path covering $Q_j$ so $k_j\geq pc(Q_j)$. 
    For every $j=1,2,\dots ,s$ we use the path contraction of $\mathcal{F}_j$ and delete the remmaining arcs in $Q_j$ results as argued in the proof of \autoref{thm:qhcycle} in a extended semicomplete digraph $S'$ having at least $pc(Q_j)$ vertices in $\overline{K}_{n_j}$ for every $j=1,2,\dots ,s$.\\
    We are now going to show that the other way.
    Suppose $S'$ contains a cycle subdigraph $\mathcal{L}$ containing $p_j\geq pc(Q_j)$ vertices of $\overline{K}_{n_j}$ for every $j=1,2,\dots ,s$.
    By \autoref{thm:5.7.7} $S'$ has a cycle $C$ containing all vertices of the cycle subdigraph $\mathcal{L}$. $V(C)=V(\mathcal{L})$. 
    Hence there is a cycle $C$ covering $p_j$ vertices of each $Q_j$. Since $p_j\geq pc(Q_j)$ there is a $p_j$-path factor in $Q_j$ for every $j=1,2,\dots ,s$.
    We then replace the vertices of $p_j$ of $\overline{K}_{n_j}$ with the $p_j$-path factor in $C$, we do this for all $j=1,2,\dots ,s$ constructing $C'$. Since replacing a vertex in a cycle with a path is still a cycle we have that $C'$ is a cycle and is the hamilton cycle of $D$. 
\end{proof}

We know that the canonical decomposition of a quasi-transitive digraph can be found in polynomial time. 
We can also find the Hamilton cycle in a quasi-transitive digraph in polynomial time, but also verify if it does not exists for the given graph. This result was proved by Gutin. And idear behind it comes from the proof \autoref{thm:notjetstatet} finding the longest cycle in an extended semicomplete digraph from a cycle subdigraph takes time $O(n^3)$ and with the extra operation of finding the coresponding $p_i$-path cover of the houses $Q_i$ we see this is the case.

\begin{thm}~\cite{banggutin94}
    There is an $O(n^4)$ algorithm which, given a quasi-transitive digraph $D$, either returns a Hamiltonian cycle in $D$ or verifies that no such cycle exists.
\end{thm}




