Finding a hamilton cycle in a digraph is a well know problem, but here is a shortly explanaition of what that is.
When we define what a hamiltonian digraph is we first have to explain what a hamilton cycle is. A hamilton cycle is a directed cycle $C_H$ in a digraph that contains(pass by) every vertex in the digraph $\forall v\in V(D), v$ is in $C_H$.\\
\begin{definition}
    A Hamiltonian digraph is a graph containing a hamilton cycle 
\end{definition} 
We can also define digraphs called traceable
\begin{definition}
    A traceable digraph is a digraph containing a hamilton path
\end{definition}
A hamilton path is a path containing all vertices of the digraph.\\
The problems that is considered \textcolor{red}{NP-Hard} is finding out whether an arbitrary digraph is trecable or hamiltonian. We are going to show that hamilton cycle problem is \textcolor{red}{NP-Hard} by reducing it to a problem we know is. Then we are going to show that if we know that a digraph is traceable it takes polynomial time to figure out wheter it is hamiltonian too, making the traceable problem \textcolor{red}{NP-Hard} too. Because if the you in polynomial time could figure out wheter a arbitrary digraph is traceable you know that if it is not, it is defenatly not hamiltonian. And if it is you can in polynomial time figure out if it is hamiltonian, making the hamton cycle problem a polynomial time solution problem (not \textcolor{red}{NP-Hard}). 